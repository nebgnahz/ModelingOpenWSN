\section{Background on OpenWSN}
\label{sec:background-openwsn}

In this section, we provide a brief overview of OpenWSN protocol. Interested readers should read \cite{watteyne2012openwsn, IEEE802.15.4e} and visit OpenWSN website\footnote{https://openwsn.atlassian.net/wiki/} for more information. 

{\bf Slotframe:} IEEE802.15.4e defines a {\em slotframe} structure. A slotframe is a group of slots which repeat over time. Current OpenWSN implementation uses 10ms for each slot. Inside each slot, the node follows a {\em schedule} that tells whether the node should transmit, receive or sleep. The construction of such schedule if out of the scope of IEEE802.15.4e.

{\bf Time Synchronization:} When a node joins a network, it aligns the slot boundary with other nodes (slot synchronization) and set its Absolute Slot Number (ASN) to the network ASN (ASN synchronization). In this way, all nodes will fire at the same time and follow a pre-defined schedule. 

{\bf Resynchronization:} Since the clock on the embedded platforms is not perfect (typical clock drift is around 10 parts-per-million (ppm)). So resynchronization is needed. In OpenWSN, it's performed whenever a packet transmission happens. Each node pick the neighbor that is closer to the root of the network as time master, and the synchronization is always achieved by aligning the its next slot boundary to the master's. When there is no packet transmission, a {\em KeepAlive} packet is generated every 30s.

{\bf OpenWSN State Machine:} In each slot, given the schedule, the node follows a state machine specification for transmitting or receiving packets. The state machine also defines how the synchronization works. We defer the discussion here because this will be the main focus of our modeling in the next section.

{\bf Packet format:} There are three types of packet: \texttt{ADV}, \texttt{DATA}, \texttt{ACK}. The advertisement packet \texttt{ADV} is used to broadcast the network information (such as ASN) such that new node can join the network. The \texttt{DATA} packet encapsulate the upper layer payload. The \texttt{ACK} packet is more than an ackowledgement of the data packet. It also contains the time correction such that other nodes can perform resynchronization if needed.


%%% Local Variables: 
%%% mode: latex
%%% TeX-master: "ee219d"
%%% End: 
