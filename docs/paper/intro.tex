\section{Introduction}
\label{sec:introduction}

With the emerging {\em Internet of Things} paradigm, standardization progress has been made to unify the communication between low-cost, low-speed ubiquitous wireless devices. In particular, IEEE 802.15.4 \cite{IEEE802.15.4} was defined to specify the physical (PHY) layer and media access control (MAC) for low-rate wireless personal area networks (LR-WPANs). Though not reliable and deterministic, the main MAC strategy is still carrier sense multiple access with collision avoidance (CSMA/CA). A new IEEE802.15.4e Time-Slotted Channel Hopping (TSCH) standard \cite{IEEE802.15.4e}, which achieves high reliability through frequency agility (channel hopping) and low-power through time synchronization is under development. The OpenWSN \cite{watteyne2012openwsn} effort at Berkeley aims to provide an open-source implementation of the complete IEEE802.15.4e protocol stack on a variety of software and hardware platforms. 

In the OpenWSN project, researchers have developed OpenSim and OpenVisualizer, which enable simulating and visualizing an OpenWSN network without the need for physical devices. Though this has greatly facilitated the development of OpenWSN-oriented applications, their simulator works by compiling the firmware and run it on a regular PC. Such a simulator\footnote{ ``emulator'' is a better term.} is tremendously useful for debugging the firmware code since you can ``freeze'' the execution and inspect variables. But it's not intended for large-scale simulations and network characteristic analysis. From the original paper \cite{watteyne2012openwsn}, authors stated that the goal of OpenSim is to demonstrate that \paperquote{it is possible to build the OpenWSN stack and applications, and emulate a full network on Windows or Linux}. From one of the discussions with the first author, we know that \paperquote{with 4 nodes, you are simulating at the speed of 1X [of the real-time]}. Therefore, we need a more systematical way of modeling OpenWSN protocol for both understanding and assessing various characteristics of the network. 

In this project, we model the OpenWSN protocol using Ptolemy II \cite{PtolemyVol1:04, davis1999overview}. Though many other platforms \cite{mccanne1995ns, varga2001omnet++} exist for network simulation, we pick Ptolemy because it has well-defined models of computation and many useful features (including multiform time, an existing wireless domain, etc.) for modeling such distributed systems. Also, there is a growing community that is using Ptolemy to build wireless sensor network applications. The Ptolemy OpenWSN (PtOWSN) implementation would serve as the platform for them. 

%% In the rest of the paper, we start by describing our modeling strategy along with some OpenWSN procotol specifications. Then we present our preliminary study of a multi-hop wireless sensor network using this model. 

%%% Local Variables: 
%%% mode: latex
%%% TeX-master: "ee219d"
%%% End: 
