\documentclass[9pt,twocolumn]{article} 
\usepackage{simpleConference}
\usepackage{subfigure}
\usepackage{times}
\usepackage{graphicx}
\usepackage{amssymb}
\usepackage{url,hyperref}
\usepackage{footmisc}

% set up tight list spacing
\usepackage{enumitem} 
\setlist{nolistsep,nosep}

% for toggles
\usepackage{etoolbox}

\newcommand {\paperquote}[1]{\em ``#1''\normalfont}

% CHANGE FROM TOGGLE TRUE TO TOGGLE FALSE TO HIDE COMMENTS
\newtoggle{comments}
%\toggletrue{comments}
\toggletrue{comments}

% Comment region command (from Wesley Willett)
\usepackage[usenames]{color}
\usepackage[usenames,dvipsnames]{xcolor}
\iftoggle{comments} {
  %if we want to show comments
  %% ============================================================
  %% add your names here
  %% possible colors include:
  %%   Orange, magenta, NavyBlue, violet, BrickRed, OliveGreen
  %% ============================================================
  \newcommand {\ben}[1]{{\color{BrickRed}\bf{BZ: #1}\normalfont}}
  \newcommand {\ai}[1]{{\color{OliveGreen}\bf{YX: #1}\normalfont}}
}{
  %if we don't want to show comments
  \newcommand {\ben}[1]{}
  \newcommand {\ai}[1]{}
}


%%% Local Variables: 
%%% mode: latex
%%% TeX-master: "ee219d"
%%% End: 


\begin{document}

\title{Project PtOWSN: Modeling OpenWSN MAC Layer using Ptolemy II}

\author{
  Antonio Iannopollo, Ben Zhang\\
  EECS, UC Berkeley\\
  \{antonio, benzh\}@eecs.berkeley.edu
}

\maketitle
\thispagestyle{empty}

\begin{abstract}
The emerging of the {\em Internet of Things} paradigm poses new challenges on the side of new application development. The usual and comfortable assumption of unlimited available resources is not only no longer valid, but completely reversed. Planning the distribution grade and the energy footprint of an application running on multiple low power nodes is a key factor for a successful result.
In this project, we describe the modeling and implementation the OpenWSN MAC layer with Ptolemy II. This network layer implements the IEEE802.15.4e standard, a variant of IEEE802.15.4 in which a set of time synchronized nodes communicate each other according to a predefined schedule. 
The result of our work is a fine grained, executable model able to provide precise information about power consumption of a network of nodes running arbitrary applications. We also provide a preliminary study of a simple multi-hop data transmission application.
\end{abstract}

\section{Introduction}
\label{sec:introduction}

With the \textbf{Internet of Things} enabling great applications including energy-aware homes and real-time asset tracking, we are seeing new standards defined that unifies the communication between low-cost, low-speed ubiquitous wireless devices. Specifically, IEEE 802.15.4 \cite{lan2003part} was defined in 2003 to specify the physical (PHY) layer and media access control (MAC) for low-rate wireless personal area networks (LR-WPANs). Though promising, the main MAC strategy is still CSMA/CA which is not appropriate for energy-constrained wireless sensor nodes. IEEE802.15.4e \cite{IEEE802.15.4e} is later proposed as an amendment that adopts channel hopping strategy to improve support for the industrial markets increases, robustness against external interference and persistent multi-path fading. 

OpenWSN project \cite{watteyne2012openwsn}, developed at Berkeley, is to provide an open-source implementation of the complete IEEE802.15.4e protocol stack on a variety of software and hardware platforms.

In this project, we model the OpenWSN protocol using Ptolemy \cite{davis1999overview, PtolemyVol1:04}. 

%%% Local Variables: 
%%% mode: latex
%%% TeX-master: "ee219d"
%%% End: 

\section{Background on OpenWSN}
\label{sec:background-openwsn}


\begin{figure*}[!htb]
\centering
\includegraphics[width=1\textwidth]{figures/PaperOpenWSNNode}
\caption{\small An example application of three nodes ({\em left}) and the simplified network stack ({\em right}).}
\label{fig:OpenWSNNode}
\end{figure*}

In this section, we provide a brief overview of the OpenWSN protocol stack. Interested readers can read \cite{watteyne2012openwsn, IEEE802.15.4e} and visit OpenWSN website\footnote{\label{note:openWSN}https://openwsn.atlassian.net/wiki/} for more information. 

{\bf Slotframe:} IEEE802.15.4e defines a {\em slotframe} structure. A slotframe is a group of temporal slots which is repeated over time. Current OpenWSN implementation uses 15ms slots. A {\em schedule} tells the node whether it should transmit, receive or sleep in a particular slot. The construction of such schedule if out of the scope of IEEE802.15.4e.

{\bf Time Synchronization:} When a node joins a network, it aligns its slot boundary with other nodes (slot synchronization) and set its Absolute Slot Number (ASN) to the network ASN (ASN synchronization). In this way, all nodes will enter in a new slot at the same time (following their pre-defined schedule). 

{\bf Re-synchronization:} Since the clock on the embedded platforms is not perfect (typical clock drift is around 10 parts-per-million (ppm)), a re-synchronization mechanism is needed. OpenWSN performs re-synchronization whenever a packet transmission happens. Each node picks as reference the neighbor that is closer to the root of the network as its time master, and the synchronization is achieved by aligning its next slot boundary to the master's. When there is no packet transmission, a {\em KeepAlive} packet is generated (every 30 seconds is a typical value).

{\bf OpenWSN State Machine:} In each slot, given the schedule, the node behaves according to a state machine. This state machine defines how the synchronization is reached and maintained, as well as transmission and reception. We defer the discussion here because this will be the main focus of our modeling work described in the next section.

{\bf Packet format:} the OpenWSN stack defines several types of packet. In our work, we consider three of them, related to the MAC layer: \texttt{ADV}, \texttt{DATA}, \texttt{ACK}. The advertisement packet \texttt{ADV} is used to broadcast the network information (such as ASN) such that new node can join the network. The \texttt{DATA} packet encapsulates the upper layer payload. The \texttt{ACK} packet is used to acknowledge a data packet. It also contains time correction information such that the other node can perform re-synchronization (if needed).


%%% Local Variables: 
%%% mode: latex
%%% TeX-master: "ee219d"
%%% End: 


\section{Modeling in Ptolemy}
\label{sec:modeling-ptolemy}

In this section, we present the strategy we followed to implement in Ptolemy the MAC layer of OpenWSN\footnote{Though the model in each figure is vector graphics, and it is possible to zoom in and read all the details, we  suggest the reader to open the attached Ptolemy model for further information}.

Our focus has been the modeling of the TSCH state machine. However, to obtain a runnable model we have also modeled a simple physical radio and abstracted the higher network layers as a single application layer. Figure~\ref{fig:OpenWSNNode} shows our stack of actors. We can see how the information flow proceed from the physical channel to the application layer and \emph{vice versa}. 

Taking a closer look, we defined each actor as follows:

{\bf PhysicalLayer:} This actor provides the interface to and from the physical channel and the rest of the network. Ports \texttt{fromPhysical}, \texttt{toPhysical}, \texttt{fromMAC} and \texttt{toMAC} are used to exchange data both during transmission and reception.
To simulate a realistic behavior, in which a packet needs a certain time to be received/sent, the \texttt{PhysicalLayer} actor adds delays whenever a packet needs to be forwarded in both directions. Ports \texttt{startOfFrame} and \texttt{endOfFrame} indicate to the MAC layer the beginning and the ending of a transmission/reception performed by the radio. 

{\bf AppLayer:} This actor represents the application that runs in each node. From a general point of view, the definition of the node behavior is this actor's responsibility. It communicates with the MAC layer through In this work, since our focus has been the MAC layer, we haven't implemented complex application models, leaving this task as a future work. \ai {I am not really sure about this paragraph}

\begin{figure*}[t]
\centering
\includegraphics[width=0.9\textwidth]{figures/PaperTSCHStateMachine}
\caption{The state machine with refinements of TSCH protocol.}
\label{fig:TSCHSM}
\end{figure*}

{\bf MACLayer:} In our model, this is the most important actor. Figure~\ref{fig:TSCHSM} gives an immediate yet detailed overview of its internal structure.
Actors \texttt{packetQueueManager} and \texttt{PacketProcessor} are used to handle packets coming from the application and the physical layers. In the first case, since the application sends packets arbitrarily, we need a queue management actor to store packets till the system is allowed to send them. In the second case, since a packet is received from the physical layer only when the MAC layer is in the reception phase, we just need an actor to extract information as the payload, the sender address, the validity of the packet, etc.
\texttt{scheduler} actor is responsible to communicate to the TSCH state machine what is the current slot ASN and when each slot starts. In our model, according to the actual OpenWSN implementation\footref{note:openWSN}, the schedule is predefined for each node. 

The \texttt{TSCHStateMachine} is implementated as a Ptolemy \emph{modal mode actor} and consists of five main states, which describe the node activity: \texttt{init}, \texttt{SLEEP}, \texttt{synchronization}, \texttt{tx} and \texttt{rx}. 

During the initialization phase, all protocol related parameters are reset. When in the \texttt{synchronization} refinement, the node keeps listening for \texttt{ADV} packet and performs slot synchronization and ASN synchronization. Once the synchronization is done, it goes back to \texttt{SLEEP} state.
The scheduler decides node actions: 
\begin{itemize}
\item If the current slot is an \texttt{ADV} slot, then the node enters the \texttt{tx} state and sends an \texttt{ADV} packet. 
\item If it is a \texttt{TX} or \texttt{RXTX} slot and the node has data to send (there are packets queued at \texttt{packetQueueManager} actor), it will enter \texttt{tx} state and send the data packet. 
\item If it's \texttt{RX} slot, or it's \texttt{RXTX} slot but the application doesn't have data to send, the node will enter \texttt{rx} state and listen for packets.
\end{itemize}

In both \texttt{tx} and \texttt{rx} states, there is a refinement which captures the complicated state transitions defined by the OpenWSN MAC layer specification (see Figure~\ref{fig:TSCHSM} on the right). For example, when the node is sending a packet, the \texttt{TSCH} state machine makes a transition from \texttt{SLEEP} to \texttt{tx} and then the associate refinement is enabled. Once in the refinement, the state machine follows a certain number of steps according to precise time constraints. Some states have additional refinements that are used to perform very specific actions. These states include \texttt{TXDATA}, where a packet is actually sent through the radio, or \texttt{TXPROC}, in which the re-synchronization information is extracted from the received \texttt{ACK} packet (see Figure~\ref{fig:timeCorrection}). Given our space constraints, we refer the reader to our Ptolemy model for more details on state transition behavior.

The \texttt{synchronization} state (in the high level \texttt{TSCH} state machine) is only entered when the node just joined the network or when it is not synchronized to the rest of the network anymore. 

In general, for every received packet (both \texttt{DATA} or \texttt{ACK} packet), the node will always capture the reception time (Figure~\ref{fig:timeCorrection} {\em top}). If the packet is from its time master (the node contacted to join the network), it will calculate the time discrepancy and use it for its own re-synchronization (Figure~\ref{fig:timeCorrection} {\em middle}). If the packet is from a time slave, then the node will still compute the time difference and send the correction through the \texttt{ACK} packet (Figure~\ref{fig:timeCorrection} {\em down}).

\begin{figure}[t]
\centering
\includegraphics[width=0.9\columnwidth]{figures/PaperReSynchronization}
\caption{Modeling the re-synchronization.}
\label{fig:timeCorrection}
\end{figure}


\subsection{Energy Consumption Modeling}
\label{sec:energy}

In \cite{vilajosana2013realistic}, authors measured power consumption values  for several radio chips and microcontrollers running the OpenWSN stack. Moreover, they annotated each state of the OpenWSN \texttt{TSCH} state machine with the correspondent state of the radio and of the microcontroller.
Similarly, we used those radio and microcontroller states to label our \texttt{TSCH} state machine.
For each node, at runtime, an additional actor compute the power consumption value considering the \texttt{TSCH} state machine annotations and the amount of time the MAC layer spends in each state.
For additional details, we suggests to the reader to check the attached model and demos.



%%% Local Variables: 
%%% mode: latex
%%% TeX-master: "ee219d"
%%% End: 

\section{Case Study}
\label{sec:case-study}

We consider a multi-hop network which has a chain of OpenWSN nodes (see Figure~\ref{fig:multihop}), and the {\em NodeId} (assigned as an integer) is increasing from left to right. 
\begin{figure}[t]
\centering
\includegraphics[width=1\columnwidth]{figures/PaperDemoPtolemy}
\caption{A multi-hop OpenWSN network.}
\label{fig:multihop}
\end{figure}
Their schedules are constructed in the following way\footnote{Interestingly, we are inspired by Gustove function to came up with such a schedule so that at any given time, a node is communicating only with one of its neighbors such that there will not be hidden terminal problem if time synchronization is achieved.}:

\begin{tabular}{ l | c | c | c | c | c }
  \hline                       
  NodeId & \multicolumn{5}{c}{Schedules} \\
  \hline
  $3i+0$ & \texttt{ADV} & \texttt{TX} & \texttt{RX} & \texttt{OFF} & $k \times \texttt{OFF}$ \\
  $3i+1$ & \texttt{ADV} & \texttt{RX} & \texttt{OFF} & \texttt{TX} & $k \times \texttt{OFF}$ \\
  $3i+2$ & \texttt{ADV} & \texttt{OFF} & \texttt{TX} & \texttt{RX} & $k \times \texttt{OFF}$ \\
  \hline  
\end{tabular}
where $i \in \mathbb{N}$ and $k$ is specified to control the duty cycle of each node (the number of \texttt{OFF} slots in the schedule; higher $k$ indicates lower duty cycle). Intuitively, when $k$ is small, nodes can receive \texttt{ADV} packet in a relatively timely fashion. This helps especially when they have lost synchronization. However, in this case the price they need to pay is to spend more time in \texttt{TX} and \texttt{RX} states, which potentially increases the power consumption. On the other hand, if $k$ is large, nodes that has lost synchronization will have to wait longer, leading to an increased energy consumption because of turning radio on and listening all the time. There might exist a {\em sweet point} of the duty cycle in this schedule pattern such that the overall energy consumption can be minimized. We leave a formal study as future work, and only focus on cases when $k = 0, 144, 306$ in this report.

\begin{figure}
\hfill
\subfigure[For each node, the time it first get synchronized (the time it joins the network).]{ \includegraphics[width=.45\linewidth]{figures/synch_time_sched}}
\hfill
\subfigure[For each node, the measured power consumption after then network is on for 20 seconds.]{ \includegraphics[width=.45\linewidth]{figures/power_cons}}
\hfill
\caption{Performance evaluation for different schedules in multi-hop network. Schedule A, B, C are corresponding to $k = 0, 144, 306$.}
\label{fig:evaluation}
\end{figure}

We present the results of a network that have 6 chained nodes\footnote{The first synchronization figure was obtained with a simulation of 20 nodes. Given that the linear relationship is clear, we present the graph with 6 nodes for consistency with the power consumption results.} under three different schedules in Figure~\ref{fig:evaluation}; and use two metrics to evaluate the schedule. The first is the time of first synchronization event for each node in the network. This metric gives us an idea of how \emph{reactive} the network is. The second is the power consumption of each node in the network for a given period of execution. This helps to evaluate the overall lifetime of a network, which correlates to the network {\em durability}.

From the left figure, we see that for different nodes the first time it is synchronized is proportional to its relative distance to the root, which is within expection. For different schedules, if we fix a specific node, when $k$ is small (Schedule A), each node has a higher duty cycle and thus it waits for a shorter time for synchronization. 

From the power consumption figure ({\em right}), we have found that if the duty cycle is high ($k$ is small), then most nodes are busy \texttt{TX} and \texttt{RX}, and the consumptions are roughly the same across different nodes. When the duty cycle is low, then the nodes that are farther from the root tends to stay more in \texttt{synchronization} state which turns on the radio for listening \texttt{ADV} packet. The interesting part is the schedule with a moderate duty cycle such that the time synchronization property is balanced with data transmission schedules, leading to an overall smaller amount of energy consumption. 

%%% Local Variables: 
%%% mode: latex
%%% TeX-master: "ee219d"
%%% End: 


\section{Conclusion}
\label{sec:conclusion}

Work is proceeding smoothly following our previous roadmap (with minor changes). We are extending our model according to an iterative approach, which allows us to capture feedback and fix issues while improving its functionality. By the end of the month, we plan to simulate inter-node communication and time synchronization, tag each execution with a specific power consumption index.



%%% Local Variables: 
%%% mode: latex
%%% TeX-master: "report"
%%% End: 


\bibliographystyle{abbrv}
{\footnotesize \bibliography{ee219d}}

\appendix
\section{Demos}

We have constructed two demos that can serve the purpose of understanding the time synchronization and multi-hop data transmission in OpenWSN protocol. They are named \verb+demo_multihop_dict_nopower.xml+(see Figure~\ref{fig:multihop}) and \verb+demo_sync.xml+(see Figure~\ref{fig:timesync}) in the accompanying folder with this report. Readers are welcome to open it and run it with Ptolemy II. 

\begin{figure}[t]
\centering
\includegraphics[width=0.9\columnwidth]{figures/PaperDemoSync}
\caption{\small Ptolemy demo that shows the action of time synchronization.}
\label{fig:timesync}
\end{figure}


%%% Local Variables: 
%%% mode: latex
%%% TeX-master: "ee219d"
%%% End: 


\end{document}
