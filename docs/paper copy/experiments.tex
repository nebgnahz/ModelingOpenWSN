\section{Introduction}
\label{sec:introduction}

With the {\em Internet of Things} enabling great applications including energy-aware homes and real-time asset tracking, standardization progress has also been made to unify the communication between low-cost, low-speed ubiquitous wireless devices. Specifically, IEEE 802.15.4 \cite{IEEE802.15.4} was defined to specify the physical (PHY) layer and media access control (MAC) for low-rate wireless personal area networks (LR-WPANs). Though promising, the main MAC strategy is still carrier sense multiple access with collision avoidance (CSMA/CA), which is not reliable and determinstic. A new IEEE802.15.4e Time-Slotted Channel Hopping (TSCH) standard \cite{IEEE802.15.4e}, which achieves high reliability through frequency agility (channel hopping) and low-power through time synchronization is under development. The OpenWSN \cite{watteyne2012openwsn} effort at Berkeley aims to provide an open-source implementation of the complete IEEE802.15.4e protocol stack on a variety of software and hardware platforms. 

In OpenWSN project, researchers have developed OpenSim and OpenVisualizer that enables simulating and visualizing an OpenWSN network without physical devices. Though this has greatly facilitated the development of application on OpenWSN protocol stack, their simulator works by compiling the firmware and run it on a regular PC. Such a simulator\footnote{ ``emulator'' is a better term.} is tremendously useful for debugging the firmware code since you can ``freeze'' the execution and inspect variables. But it's not intended for large-scale simulation and network characteristic analysis. From the original paper \cite{watteyne2012openwsn}, they stated that the goal of OpenSim is to demonstrate that \paperquote{it is possible to build the OpenWSN stack and applications, and emulate a full network on Windows or Linux}. From one of the discussions with the first author, he mentioned that \paperquote{with 4 nodes, you are simulating at the speed of 1X [of the realtime]}. Therefore, we need a more systematical way of modeling OpenWSN protocol for both understanding and assessing various characteristics of the network. 

In this project, we model the OpenWSN protocol using Ptolemy II \cite{PtolemyVol1:04, davis1999overview}. Though many other platforms \cite{mccanne1995ns, varga2001omnet++} exist for network simulation, we pick Ptolemy because it has well-defined models of computation and many useful features (including multiform time, an existing wireless domain, etc.) for modeling such distributed systems. Also there are increasing need of using Ptolemy to build wireless sensor network applications. The Ptolemy OpenWSN (PtOWSN) implementation would serve as the platform for them. 

%% In the rest of the paper, we start by describing our modeling strategy along with some OpenWSN procotol specifications. Then we present our preliminary study of a multi-hop wireless sensor network using this model. 

%%% Local Variables: 
%%% mode: latex
%%% TeX-master: "ee219d"
%%% End: 
